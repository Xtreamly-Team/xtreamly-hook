\documentclass[12pt]{article}

\usepackage{amsmath,amssymb,amsthm}
\usepackage{graphicx}
\usepackage[margin=1in]{geometry}

\begin{document}

\title{\textbf{Hedging Strategies with Perpetuals for Uniswap V4 Liquidity Positions}}
\author{%
  \textbf{Xtreamly}%
}
\date{\today}
\maketitle

\begin{abstract}
This document examines the use of perpetual futures to hedge Uniswap V4 concentrated liquidity positions. By combining long and short perpetual strategies, liquidity providers can reduce downside risk and participate in further price appreciation. The impact of leverage, funding costs, and liquidation risks is discussed, highlighting the importance of optimization and backtesting for effective risk management.
\end{abstract}

\section{Value of a Uniswap V4 Position (Ignoring Fees)}
\label{sec:uniswaplp}

Consider a Uniswap V4 position providing liquidity between the lower price bound and upper bound. For concreteness, suppose we are providing liquidity in the ETH/USDT pool:
\begin{itemize}
    \item \textbf{ETH (token0)} is the volatile asset.
    \item \textbf{USDT (token1)} is the stable asset (or stable-ish).
\end{itemize}

\subsection{Notation}
We define the following variables:
\begin{itemize}
    \item $P_L$: The lower bound price of the LP position (USDT per ETH).
    \item $P_U$: The upper bound price of the LP position (USDT per ETH).
    \item $P$: The current market price of ETH in USDT.
    \item $L$: The \emph{constant} liquidity associated with the position.
    \item $X_0$: The amount of ETH (token0) held at the time the position is set, related to $P_U$ and $P_L$.
    \item $Y_0$: The amount of USDT (token1) held at the time the position is set, also related to $P_U$ and $P_L$.
\end{itemize}

\begin{center}
\textbf{Ignoring fees}, the LP payoff depends on which region the price $P$ falls into:
\end{center}

\subsection{Three Cases for the LP Value}

\textbf{Case 1: Within Range} $P_L \le P \le P_U$ 

\medskip

When the market price $P$ lies within the liquidity provider's chosen range $[P_L, P_U]$, the position consists of a mix of ETH and USDT. The exact amounts of each asset are determined by Uniswap's liquidity formula.

\medskip

At any price $P$ in this range:
\[
X(P) = L \biggl(\frac{1}{\sqrt{P}} - \frac{1}{\sqrt{P_U}}\biggr),
\qquad
Y(P) = L \bigl(\sqrt{P} - \sqrt{P_L}\bigr).
\]
where:
\begin{itemize}
    \item $X(P)$ represents the amount of ETH (token0) held at price $P$.
    \item $Y(P)$ represents the amount of USDT (token1) held at price $P$.
    \item $L$ is the total liquidity provided to the range.
\end{itemize}

The total value of the LP position at price $P$ is then:
\[
V(P) = X(P) \cdot P \;+\; Y(P).
\]

Expanding:
\[
V(P) = L \biggl(\frac{1}{\sqrt{P}} - \frac{1}{\sqrt{P_U}}\biggr) P
\;+\;
L \bigl(\sqrt{P} - \sqrt{P_L}\bigr),
\quad \text{for } P_L \le P \le P_U.
\]

\medskip

\textbf{Case 2: Below Range} $P < P_L$.

\medskip

When the price drops below $P_L$, the liquidity position has been fully converted into ETH (token0). This happens because as price decreases, the Uniswap AMM mechanism sells all available USDT in exchange for ETH.

\medskip

The ETH amount is locked at the value it had at $P_L$, which is:
\[
X(P_L) = L \biggl(\frac{1}{\sqrt{P_L}} - \frac{1}{\sqrt{P_U}}\biggr).
\]
Since the LP now holds only ETH, the total position value in USD is simply the ETH amount multiplied by the current price $P$:
\[
V(P) = X(P_L) \cdot P.
\]
Expanding:
\[
V(P) = L \biggl(\frac{1}{\sqrt{P_L}} - \frac{1}{\sqrt{P_U}}\biggr) P,
\quad \text{for } P < P_L.
\]

\medskip

\textbf{Case 3: Above Range} $P > P_U$.

\medskip

When the price moves above $P_U$, the liquidity position has been fully converted into USDT (token1). This happens because as price increases, the Uniswap AMM mechanism sells all available ETH in exchange for USDT.

\medskip

The USDT amount is locked at the value it had at $P_U$, which is:
\[
Y(P_U) = L \bigl(\sqrt{P_U} - \sqrt{P_L}\bigr).
\]
Since the LP now holds only USDT, the total position value remains constant:
\[
V(P) = Y(P_U).
\]
Expanding:
\[
V(P) = L \bigl(\sqrt{P_U} - \sqrt{P_L}\bigr),
\quad \text{for } P > P_U.
\]

\medskip

\paragraph{Final Piecewise Formula.}
Bringing everything together, the complete value function for the LP position is:
\[
V(P) =
\begin{cases}
L \biggl(\frac{1}{\sqrt{P_L}} - \frac{1}{\sqrt{P_U}}\biggr) P, & P < P_L,\\[8pt]
L \Bigl(\bigl(\tfrac{1}{\sqrt{P}} - \tfrac{1}{\sqrt{P_U}}\bigr) P + \bigl(\sqrt{P} - \sqrt{P_L}\bigr)\Bigr), & P_L \le P \le P_U,\\[10pt]
L \bigl(\sqrt{P_U} - \sqrt{P_L}\bigr), & P > P_U.
\end{cases}
\]

\begin{figure}[htb]
    \centering
    \includegraphics[width=0.7\textwidth]{images/CLposition.png}
    \caption{illustrating the piecewise value of a Uniswap V4 position (ignoring fees).}
    \label{fig:CLposition}
\end{figure}

\newpage

\section{Introducing a Perpetual Hedge}
\label{sec:perphedge}

We consider a perpetual futures contract to hedge ETH/USDT. Let:
\begin{itemize}
    \item $M$: The margin (in USDT) posted for the perpetual position.
    \item $\alpha$: The \emph{notional} size of the perp (in ETH units).
    \item $P_0$: The entry price (USDT per ETH).
    \item $P_{\mathrm{liq}}$: The \emph{liquidation price}, where margin is depleted and lost.\footnote{
      In practice, exchanges might have partial liquidations or maintenance margins, but here we simplify: once $P$ hits $P_{\mathrm{liq}}$, the entire margin $M$ is lost.}
\end{itemize}

\paragraph{Short Perpetual.}  
Suppose we \emph{short} at price $P_0$. The PnL from the short is $\alpha\,(P_0 - P)$. We post margin $M$ to cover losses if $P$ rises. We define $P_{\mathrm{liq}}$ so that when $P = P_{\mathrm{liq}}$, our margin is fully exhausted:
\[
M + \alpha\,(P_0 - P_{\mathrm{liq}}) = 0
\quad\Longrightarrow\quad
P_{\mathrm{liq}} = P_0 + \frac{M}{\alpha}.
\]
Hence, our \textbf{short} payoff is
\[
V_{\mathrm{short}}(P) = 
\begin{cases}
M + \alpha\,(P_0 - P), & P < P_{\mathrm{liq}},\\
-M, & P \ge P_{\mathrm{liq}}.
\end{cases}
\]

\begin{figure}[htb]
    \centering
    \includegraphics[width=0.6\textwidth]{images/short.png}
    \caption{Piecewise payoff for the Short Perpetual, showing the drop to $-M$ at liquidation.}
    \label{fig:short}
\end{figure}

\paragraph{Long Perpetual.}  
If we \emph{long} at $P_0$, the PnL is $\alpha\,(P - P_0)$. The margin $M$ is depleted if $P$ falls by enough. Setting $P = P_{\mathrm{liq}}$ where margin is exhausted:
\[
M + \alpha\,(P_{\mathrm{liq}} - P_0) = 0
\quad\Longrightarrow\quad
P_{\mathrm{liq}} = P_0 - \frac{M}{\alpha}.
\]
Hence, our \textbf{long} payoff is
\[
V_{\mathrm{long}}(P) = 
\begin{cases}
-M, & P \le P_{\mathrm{liq}},\\
\alpha\,(P - P_0), & P > P_{\mathrm{liq}}.
\end{cases}
\]

\begin{figure}[htb]
    \centering
    \includegraphics[width=0.6\textwidth]{images/long.png}
    \caption{Piecewise payoff for the Long Perpetual, showing the drop to $-M$ at liquidation.}
    \label{fig:long}
\end{figure}

\paragraph{Interpretation.}  
At liquidation ($P = P_{\mathrm{liq}}$), the payoff \emph{jumps} from some positive level (on the linear portion) down to $-M$, reflecting full loss of the margin. This discontinuous drop represents the complete loss of collateral at the liquidation boundary.


\section{Combining LP and Perpetual}
\label{sec:fiveway}

In this section, we analyze the combination of a Uniswap V4 concentrated liquidity (CL) position with a perpetual hedge. The goal is to dynamically hedge the price exposure of the LP position while maintaining capital efficiency. \\

\medskip

\subsection{Long Perpetual Hedge with Concentrated Liquidity}

In Figure~\ref{fig:combined-long}, we illustrate the combined payoff of a long perpetual hedge with a Uniswap CL position. This approach is used when the AI expects volatility and wants to ensure exposure to price appreciation of the underlying asset (ETH in our example). Without a hedge, once the price moves above $P_U$, the LP position fully converts into USDT, leading to missed upside. By incorporating a long perpetual hedge, the strategy ensures that additional gains can be captured.

\medskip

\textbf{Key observations from Figure~\ref{fig:combined-long}:}
\begin{itemize}
    \item The Uniswap CL position alone (dashed gray line) exhibits a flat payoff above $P_U$, since all liquidity has been converted to USDT.
    \item The long perpetual hedge introduces a positive slope above the entry price, capturing further upside.
    \item If the price declines and liquidation occurs, the strategy incurs a sharp loss at $P_{\mathrm{liq}}$.
\end{itemize}

\begin{figure}[htb]
    \centering
    \includegraphics[width=0.6\textwidth]{images/combined_long.png}
    \caption{Example combined payoff: Concentrated liquidity plus a Long Perp.}
    \label{fig:combined-long}
\end{figure}

\subsection{Short Perpetual Hedge with Concentrated Liquidity}

In Figure~\ref{fig:combined-short}, we show the combined payoff of a short perpetual hedge with a Uniswap CL position. This strategy is employed when we seek to protect against downside risk. Without a hedge, if the price drops below $P_L$, the position converts entirely into ETH, exposing the LP to further losses. A short perpetual hedge compensates for these losses.

\medskip

\textbf{Key observations from Figure~\ref{fig:combined-short}:}
\begin{itemize}
    \item The Uniswap CL position alone (dashed gray line) exhibits a linear decline below $P_L$ due to full conversion into ETH.
    \item The short perpetual hedge introduces a slope below the entry price, mitigating further downside.
    \item If the price increases and liquidation occurs, the hedge incurs a sharp loss at $P_{\mathrm{liq}}$.
\end{itemize}

\medskip

\begin{figure}[htb]
    \centering
    \includegraphics[width=0.6\textwidth]{images/combined_short.png}
    \caption{Example combined payoff: Concentrated liquidity plus a Short Perp.}
    \label{fig:combined-both}
\end{figure}

Figure~\ref{fig:combined-both} illustrates the combined payoff when using both a short and long perpetual hedge alongside a Uniswap CL position. This configuration aims to provide protection on both the downside and upside, mitigating the risk of price movements that push the LP position out of range.

\medskip

\textbf{Key insights:}
\begin{itemize}
    \item The long perpetual hedge compensates for missed upside when the price surpasses $P_U$, ensuring the strategy does not forgo further gains once the LP position is fully converted to USDT.
    \item The short perpetual hedge offsets losses when the price declines below $P_L$, counteracting the exposure to ETH that would otherwise be unhedged.
    \item The combined strategy flattens the risks associated with concentrated liquidity by maintaining a more balanced exposure across price ranges.
\end{itemize}

However, this strategy is not without risks. If the price oscillates too frequently around the entry levels of the perpetual positions, liquidations can occur before meaningful profits are realized. This would result in accumulated losses, as the hedges fail to serve their intended purpose of long-term exposure management. 

\medskip

Thus, to maximize the effectiveness of this combined approach, it is crucial to:
\begin{itemize}
    \item Optimize hedge sizing to avoid excessive liquidation risk while maintaining capital efficiency.
    \item Adjust leverage cautiously, as higher leverage increases sensitivity to liquidation but also amplifies returns when successful.
    \item Backtest under different market conditions to analyze whether the perpetual hedges improve overall strategy performance or lead to excessive churn.
\end{itemize}

Overall, the combined approach offers significant downside protection and potential for additional upside capture, but requires careful risk management to prevent liquidation losses from overwhelming the benefits of the hedging strategy.

\subsection{Considerations for Hedge Optimization}

The effectiveness of this strategy depends on proper hedge allocation and leverage management:
\begin{itemize}
    \item \textbf{Collateral allocation:} The fraction of total capital allocated to the hedge determines its impact on the overall payoff.
    \item \textbf{Leverage choice:} Higher leverage increases capital efficiency but also raises funding costs and liquidation risk.
    \item \textbf{Funding rate impact:} Perpetual futures often have non-zero funding rates, which can accumulate over time, affecting profitability.
    \item \textbf{Backtesting necessity:} The interaction between CL payoffs and perpetual hedges needs to be tested across various market conditions to ensure robustness.
\end{itemize}

By systematically balancing these parameters, liquidity providers can dynamically adjust their hedge exposure, optimizing for capital efficiency while managing risk exposure.

\begin{figure}[htb]
    \centering
    \includegraphics[width=0.6\textwidth]{images/combined_both.png}
    \caption{Example combined payoff: Concentrated liquidity plus a Short and Long Perp.}
    \label{fig:combined-short}
\end{figure}

\newpage

\end{document}